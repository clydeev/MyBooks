\documentclass[12pt]{book}
\begin{document}
\title{Triumph of the Church\\(History of Heresies and its Refutation)}
\author{St. Alphonsus Marie de Liguori}
\date{}
\maketitle
\noindent{
Dublin: Published by James Duffy, 10, Wellington-Quay. 1847.
\\Printed by William Holdin, 10, Abbey-street.}

\chapter*{Translator's Preface}
THE ardent wish manifested by "the Faithful for an acquaintance with the valuable writings of ST. 
LIGUORI, induced me to undertake the Translation of his History of Heresies, one of his greatest works.
The Holy Author was induced to write this Work, to meet the numbers of infidel publications, with
which Europe was deluged in the latter half of the last century. Men’s minds were then totally unsettled;
dazzled by the glare of a false philosophy, they turned away from the light of the Gospel. The heart of the
Saint was filled with sorrow, and he laboured to avert the scourge he saw impending over the unfaithful
people. He implored the Ministers of his Sovereign to put the laws in force, preventing the introduction
of irreligious publications into the Kingdom of Naples, and he published this Work, among* others, to
prove, as he says, that the Holy Catholic Church is the only true one the Mistress of Truth the Church,
founded by Jesus Christ himself, which would last to the end of time, notwithstanding the persecutions
of the infidel, and the rebellion of her own heretical children. He dedicates the Book to the Marquis
Tanucci, the Prime Minister of the Kingdom, whom he praises for his zeal for Religion, and his vigorous
execution of the laws against the vendors of infidel publications. He brings down the History from the
days of the Apostles to his own time, concluding- with the Refutation of the Heresies of Father Berruyer.

I have added a Supplementary Chapter, giving- a succinct account of the Heretics and Fanatics of the last
eighty years. It was, at first, my intention to make it more diffuse; but, then, I considered that it would be
out of proportion with the remainder of the Work. This Book may be safely consulted, as a work of
reference: the Author constantly quotes his authorities; and the Student of Ecclesiastical History can at
once compare his statements with the sources from which he draws. In the latter portion of the Work, and
especially in that portion of it, the most interesting- to us, the History of the English Reformation, the
Student may perceive some slight variations between the original text and my translation. I have collated
the Work with the writings of modern Historians the English portion, especially with Hume and Lingard
and wherever I have seen the statements of the Holy Author not borne out by the authority of our own
Historians, I have considered it more prudent to state the facts, as they really took place; for our own
writers must naturally be supposed to be better acquainted with our History, than the foreign authorities
quoted by the Saint. The reader will also find the circumstances, and the names of the actors, when I
considered it necessary, frequently given more in detail than in the original.

In the style, I have endeavoured, as closely as the genius of our language would allow, to keep to the
original. St. Alphonsus never sought for ornament; a clear, lucid statement of facts is what he aimed at;
there is nothing inflated in his writings; he wrote for the people, and that is the principal reason, I
imagine, why not only his Devotional Works, but his Historical and Theological Writings, also, have been
in such request: but, while he wrote for the people, we are not to imagine that he did not also please the
learned. His mind was richly stored with various knowledge; he was one of the first Jurists of his day; his
Theological science elicited the express approbation of the greatest Theologian of his age Benedict XIV.;
he was not only a perfect master of his own beautiful language, but profoundly read in both Greek and
Latin literature also, and a long life constantly employed in studies, chiefly ecclesiastical, qualified him,
above any man of his time, to become an Ecclesiastical Historian, which no one should attempt unless he
be a general I might almost say a universal, scholar : so much for the Historical portion of the Work.

In the Second Part, the Refutation of Heresies, the Holy Author comprises, in a small space, a vast
amount of Theological information; in fact, there is no Heresy which cannot be refuted from it. Not alone
are the usual Heresies, which we have daily to combat such as those opposed to the Real Presence, the
Authority of the Church, the doctrine of Justification, clearly and diffusely refuted, but those abstruse
heretical opinions concerning- Grace, Free Will, the Procession of the Holy Ghost, the Mystery of the
Incarnation, and the two Natures of Christ, and so forth, are also clearly and copiously confuted; the
intricacies of Pelagianism, Calvinism, and Jansenism, are unravelled, and the true Doctrine of the Church
triumphantly vindicated. The reader will find, in general, the quotations from the Fathers in the original,
but those unacquainted with Latin will easily learn their sentiments from the text. The Scripture 
quotations are from the Douay Version.

Every Theologian will be aware of the difficulty of giving- scholastic terms in an English dress. In the
language of the Schools, the most abstract ideas, which would require a sentence to explain them in our
tongue, are most appropriately expressed by a single word; all the Romance languages, daughters of the
Latin, have very nearly the same facility, but our Northern tongue has not, I imagine, flexibility enough
for the purpose. I have, however, endeavoured, as far as I could, to preserve the very terms of the
original, knowing how easy it is to give a heterodox sense to a passage, by even the most trivial deviation
from the very expression of the writer. The Theological Student will thus, I hope, find the Work a
compact Manual of Polemic Theology; the Catholic who, while he firmly believes all that the Church
teaches, wishes to be able to give an account of the Faith that is in him, will here find it explained and
defended; while those not of the " fold," but for whom we ardently pray, that they may hear the voice of
the " one Shepherd," may see, by its attentive perusal, that they inhabit a house " built upon the sand,"
and not the house " on the rock."

They will behold the mighty tree of Faith sprung from the grain of mustard-seed planted by our
Redeemer, always flourishing always extending*, neither uprooted by the storms of persecution, nor
withered by the sun of worldly prosperity. Nay more, the very persecution the Church of God has
suffered, and is daily enduring, only extends it more and more; the Faithful, persecuted in " one city," fly
elsewhere, bearing with them the treasure of Faith, and communicating it to those among- whom they
settle, as the seeds of fertility are frequently borne on the wings of the tempest to the remote desert,
which would otherwise be cursed with perpetual barrenness. The persecution of the Church in Ireland,
for example, "has turned the desert into fruitfulness," in America, in Australia, in England itself, and the
grey mouldering ruins of our fanes on the hill sides are compensated for by the Cathedral Churches
across the ocean. The reader will see Heresy in every age, from the days of the Apostles themselves down
to our own time, rising up, and vanishing after a while, but the Church of God is always the same, her
Chief Pastors speaking with the same authority, and teaching the same doctrine to the trembling
Neophytes’ in the Catacombs, and to the C{\ae}sars on the throne of the world. Empires are broken into
fragments and perish nations die away, and are only known to the historian languages spoken by
millions disappear every thing that is man’s work dies like man; heresies, like the rest, have their rise,
their progress, their decay, but Faith alone is eternal and unchangeable, "yesterday, to-day, and the same
for ever."

\chapter*{Author's Preface}
\begin{enumerate}
\item My object in writing this work is to prove that the Roman Catholic Church is the only true one among
so many other Churches, and to show how carefully the Almighty guarded her, and brought her
victoriously through all the persecutions of her enemies. Hence, as St. Ir{\ae}neus says (Lib. 3, cap. 3, n. 2), all
should depend on the Roman Church as on their fountain and head. This is the Church founded by Jesus
Christ, and propagated by the Apostles; and although in the commencement persecuted and contradicted
by all, as the Jews said to St. Paul in Rome : " For as concerning this sect (thus they called the Church), we
know that it is gainsayed every where" (Acts, xxviii, 22); still she always remained firm, not like the other
false Churches, which in the beginning numbered many followers, but perished in the end, as we shall
see in the course of this history, when we speak of the Arians, Nestorians, Eutychians, and Pelagians; and
if any sect still reckons many followers, as the Mahometans, Lutherans, or Calvinists, it is easy to see that
they are upheld, not by the love of truth, but either by popular ignorance, or relaxation of morals. St.
Augustine says that heresies are only embraced by those who had they persevered in the faith, would be
lost by the irregularity of their lives (St. Aug. de Va. Rel. c. 8.)

\item Our Church, on the contrary, notwithstanding that she teaches her children a law opposed to the
corrupt inclinations of human nature, not only never failed in the midst of persecutions, but even gained
strength from them; as Tertullian (Apol. cap. ult.) says, the blood of martyrs is the seed of Christians, and
the more we are mown down the more numerous we become; and in the 20th chapter of the same work
he says, the kingdom of Christ and his reign is believed and he is worshipped by all nations. Pliny the
Younger confirms this in his celebrated Letter to Trajan, in which he says that in Asia the temples of the
gods were deserted, because the Christian Religion had overrun not only the cities but even the villages.

\item This, certainly, never could have taken place without the power of the Almighty, who intended to
establish in the midst of idolatry, a new religion, to destroy all the superstitions of the false religion, and
the ancient belief in a multitude of false gods adored by the Gentiles, by their ancestors, by the
magistrates, and by the emperors themselves, who made use of all their power to protect it, and still the
Christian faith was embraced by many nations who forsook a relaxed law for a hard and difficult one,
forbidding them to pamper their sensual appetites. What but the power of God could accomplish this?

\item Great as the persecutions were which the Church suffered from idolatry, still greater were those she
had to endure from the heretics which sprang from her own bosom, by means of wicked men, who, either
through pride or ambition, or the desire of sensual license, endeavoured to rend the bowels of their
parent. Heresy has been called a canker : " It spreadeth like a canker" (II. Tim. ii, 17); for as a canker
infects the whole body, so heresy infects the whole soul, the mind, the heart, the intellect, and the will. It
is also called a plague, for it not only infects the person contaminated with it, but those who associate
with him, and the fact is, that the spread of this plague in the world has injured the Church more than
idolatry, and this good mother has suffered more from her own children than from her enemies. Still she
has never perished in any of the tempests which the heretics raised against her; she appeared about to
perish at one time through the heresy of Arius, when the faith of the Council of Nice, through the
intrigues of the wicked Bishops, Valens and Ursacius, was condemned, and, as St. Jerome says, the world
groaned at finding itself Arian (1); and the Eastern Church appeared in the same danger during the time
of the heresies of Nestorius and Eutyches. But it is wonderful, and at the same time consoling, to read the
end of all those heresies, and behold the bark of the Church, which appeared completely wrecked and
sunk through the force of those persecutions, in a little while floating more gloriously and triumphantly
than before.

\item St. Paul says: " There must be heresies, that they also who are reproved may be made manifest among
you" (I. Cor. ii, 19). St. Augustine, explaining this text, says that as fire is necessary to purify silver, and
separate it from the dross, so heresies are necessary to prove the good Christians among the bad, and to
separate the true from the false doctrine. The pride of the heretics makes them presume that they know
the true faith, and that the Catholic Church is in error, but here is the mistake : our reason is not sufficient
to tell us the true faith, since the truths of Divine Faith are above reason; we should, therefore, hold by
that faith which God has revealed to his Church, and which the Church teaches, which is, as the Apostle
says, " the pillar and the ground of truth" (I. Tim. iii, 15).\\
\textbf{(1) St. Hieron. Dial, adversus Lucifer.}\\

Hence, as St. Ir{\ae}neus says, "It is necessary that all should depend on the Roman Church as their head and
fountain; all Churches should agree with this Church on account of her priority of principality, for there
the traditions delivered by the Apostles have always been preserved" (St. Iran, lib. 3, c. 3); and by the
tradition derived from the Apostles which the Church founded at Rome preserves, and the Faith
preserved by the succession of the Bishops, we confound those who through blindness or an evil
conscience draw false conclusions (Ibid). " Do you wish to know," says St. Augustine, " which is the true
Church of Christ? Count those priests who, in a regular succession have succeeded St. Peter, who is the
Rock, against which the gates of hell will not prevail" (St. Aug. in Ps. contra part Donat.) : and the holy
Doctor alleges as one of the reasons which detain him in the Catholic Church, the succession of Bishops to
the present time in the See of St. Peter" (Epis. fund, c. 4, n. 5); for in truth the uninterrupted succession
from the Apostles and disciples is characteristic of the Catholic Church, and of no other.

\item It was the will of the Almighty that the Church in which the true faith was preserved should be one,
that all the faithful might profess the one faith, but the devil, St. Cyprian says (2), invented heresies to
destroy faith, and divide unity. The enemy has caused mankind to establish many different churches, so
that each, following the faith of his own particular one, in opposition to that of others, the true faith might
be confused, and as many false faiths formed as there are different churches, or rather different
individuals. This is especially the case in England, where we see as many religions as families, and even
families themselves divided in faith, each individual following his own. St. Cyprian, then, justly says that
God has disposed that the true faith should be preserved in the Roman Church alone, so that there being
but one Church there should be but one faith and one doctrine for all the faithful. St. Optatus
Milevitanus, writing to Parmenianus, says, also : " You cannot be ignorant that the Episcopal Chair of St.
Peter was first placed in the city of Rome, in which one chair unity is observed by all" (St. Opt. I 2, cont.
Parmen.)\\
\textbf{(2) St. Cyprian de Unitate Ecclesi{\ae}.}

\item The heretics, too, boast of the unity of their Churches, but St. Augustine says that it is unity against
unity. " What unity," says the Saint, " can all those churches have which are divided from the Catholic
Church, which is the only true one; they are but as so many useless branches cut off from the Vine, the
Catholic Church, which is always firmly rooted. This is the One Holy, True, and Catholic Church,
opposing all heresies; it may be opposed, but cannot be conquered. All heresies come forth from it, like
useless shoots cut off from the vine, but it still remains firmly rooted in charity, and the gates of hell shall
not prevail against it" (St. Aug. lib. 1, de Symbol ad Cath. c. 6). St. Jerome says that the very fact of the
heretics forming a church apart from the Roman Church, is a proof, of itself, that they are followers of
error, and disciples of the devil, described by the Apostle, as " giving heed to spirits of error and doctrines
of devils" (I. Tim. iv, 1).

\item The Lutherans and Calvinists say, just as the Donatists did before them, that the Catholic Church
preserved the true faith down to a certain period some say to the third, some to the fourth, some to the
fifth century but that after that the true doctrine was corrupted, and the spouse of Christ became an
adulteress. This supposition, however, refutes itself; for, granting that the Roman Catholic Church was
the Church first founded by Christ, it could never fail, for our Saviour himself promised that the gates of
hell never should prevail against it : "I say unto you that you are Peter, and on this Rock I will build my
church, and the gates of hell shall not prevail against it" (Matt, xviii, 18). It being certain, then, that the
Roman Catholic Church was the true one, as Gerard, one of the first ministers of Luther, admits (Gerard
de Eccles. cap. 11, sec. 6) it to have been for the first five hundred years, and to have preserved the
Apostolic doctrine during that period, it follows that it must always have remained so, for the spouse of
Christ as St. Cyprian says, could never become an adulteress.

\item  The heretics, however, who, instead of learning from the Church the dogmas they should believe, wish
to teach her false and perverse dogmas of their own, say that they have the Scriptures on their side, which
are the fountain of truth, not considering, as a learned author (3) justly remarks, that it is not by reading,
but by understanding, them, that the truth can be found. Heretics of every sort avail themselves of the
Scriptures to prove their errors, but we should not interpret the Scripture according to our own private
opinions, which frequently lead us astray, but according to the teaching of the Holy Church which is
appointed the Mistress of true doctrine, and to whom God has manifested the true sense of the Divine
books. This is the Church, as the Apostle tells us, which has been appointed the pillar and the ground of
truth: \textit{\textbf{"that thou mayest know how thou oughtest to behave thyself in the house of God, which is the
Church of the living God, the pillar and the ground of truth"}} (I. Tim. iii, 15.) Hence St. Leo says that the
Catholic faith despises the errors of heretics barking against the Church, who deceived by the vanity of
worldly wisdom, have departed from the truth of the Gospel (St. Leo, Ser. 8, de Nat Dim.)

\item I think the History of Heresies is a most useful study, for it shows the truth of our Faith more pure
and resplendent, by showing how it has never changed; and if, at all times, this is useful, it must be
particularly so at present, when the most holy maxims and the principal dogmas of Religion are put in
doubt : it shows, besides, the care God always took to sustain the Church in the midst of the tempests
which were unceasingly raised against it, and the admirable manner in which all the enemies who
attacked it were confounded. The History of Heresies is also useful to preserve in us the spirit of humility
and subjection to the Church, and to make us grateful to God for giving us the grace of being born in
Christian countries; and it shows how the most learned men have fallen into the most grievous errors, by
not subjecting themselves to the Church’s teaching.\\
\textbf{(3) Danes, Gen. Temp. Nat. in Epil.}

\item I will now state my reasons for writing this Work; some may think this labour of mine superfluous,
especially as so many learned authors have written expressly and extensively the history of various
heresies, as Tertullian, St. Ir{\ae}neus, St. Epiphanius, St. Augustine, St. Vincent of Lerins, Socrates,
Sozymen, St. Philastrius, Theodoret, Nicephorus, and many others, both in ancient and modern times.
This, however, is the very reason which prompted me to write this Work; for as so many authors have
written, and so extensively, and as it is impossible for many persons either to procure so many and such 
expensive works, or to find time to read them, if they had them, I, therefore, judged it better to collect in a
small compass the commencement and the progress of all heresies, so that in a little time, and at little
expense, any one may have a sufficient knowledge of the heresies and schisms which infected the
Church. I have said in a small compass, but still, not with such brevity as some others have done, who
barely give an outline of the facts, and leave the reader dissatisfied, and ignorant of many of the most
important circumstances. I, therefore, have studied brevity; but I wish, at the same time, that my readers
may be fully informed of every notable fact connected with the rise and progress of, at all events, the
principal heresies that disturbed the Church.

\item Another reason I had for publishing this Work was, that as modern authors, who have paid most
attention to historical facts, have spoken of heresies only as a component part of Ecclesiastical History, as
Baronius, Fleury, Noel Alexander, Tillemont, Orsi, Spondanus, Raynaldus, Graveson, and others, and so
have spoken of each heresy chronologically, either in its beginning, progress, or decay, and, therefore, the
reader must turn over to different parts of the works to find out the rise, progress, and disappearance of
each heresy; I, on the contrary, give all at once the facts connected with each heresy in particular.

\item Besides, these writers have not given the Refutation of Heresies, and I give this in the second part of
the Work; I do not mean the refutation of every heresy, but only of the principal ones, as those of
Sabellius, Arius, Pelagius, Macedonius, Nestorius, Eutyches, the Monothelites, the Iconoclasts, the
Greeks, and the like. I will merely speak of the authors of other heresies of less note, and their falsity will
be apparent, either from their evident weakness, or from the proofs I bring forward against the more
celebrated heresies I have mentioned.

\item We ought, then, dear reader, unceasingly to thank our Lord for giving us the grace of being born and
brought up in the bosom of the Catholic Church. St. Francis de Sales exclaims: \textit{\textbf{ "O good God! many and
great are the benefits thou hast heaped on me, and I thank thee for them; but how shall I be ever able to
thank thee for enlightening me with thy holy Faith?"}} And writing to one of his friends, he says: \textit{\textbf{
"God! the beauty of thy holy Faith appears to me so enchanting, that I am dying with love of it, and I imagine I
ought to enshrine this precious gift in a heart all perfumed with devotion."}} St. Teresa never ceased to
thank God for having made her a daughter of the Holy Church: her consolation at the hour of death was
to cry out: \textit{\textbf{"I die a child of the Holy Church! I die a child of the Holy Church."}} We, likewise, should
never cease praising Jesus Christ for this grace bestowed on us one of the greatest conferred on us one
distinguishing us from so many millions of mankind, who are born and die among infidels and heretics :
\textit{\textbf{"He has not done in like manner to every nation"}} (Psalm cxlvii, 9). With our minds filled with gratitude
for so great a favour, we shall now see the triumph the Church has obtained through so many ages, over
so many heresies opposed to her. I wish to remark, however, before I begin, that I have written this Work
amidst the cares of my Bishoprick, so that I could not give a critical examination, many times, to the facts
I state, and, in such case, I give the various opinions of different authors, without deciding myself on one
side or the other. I have endeavoured, however, to collect all that could be found in the most correct and
notable writers on the subject; but it is not impossible that some learned persons may be better
acquainted with some facts than I am.
\end{enumerate}

\chapter{Heresies of the First Century}
\section{Simon Magus}
Simon Magus (1), the first heretic who disturbed the Church, was born in a part of Samaria called
Githon or Gitthis. He was called Magus, or the Magician, because he made use of spells to deceive the
multitude; and hence he acquired among his countrymen the extraordinary name of " The Great Power
of God" (Acts, viii, 1 0). " This man is the power of God which is called great." Seeing that those on whom
the Apostles Peter and John laid hands received the Holy Ghost, he offered them money to give to him
the power of communicating the Holy Ghost in like manner; and on that account the detestable crime of
selling holy things is called Simony. He went to Rome, and there was a statue erected to him in that city, a
fact which St. Justin, in his first Apology, flings in the face of the Romans : " In your royal city," he says, "
he (Simon) was esteemed a God, and a statue was erected to him in the Island of the Tyber, between the
two bridges, bearing this Latin inscription SIMONI, DEO SANCTO."\\
(1) Baron. Annal, 35, d. 23; N. Alex. Hist. Ecclesias. t. 5, c. 11, n.-l; Hermant. His. Con. 56, 1, c. 7; Van
Ranst, His. Her. n. 1.\\

Samuel Basnage, Petavius, Valesius, and many others, deny this fact; but Tillemont, Grotius, Fleury, and
Cardinal Orsi defend it, and adduce in favour of it the authority of Tertullian, St. Irenæus, St. Cyril of
Jerusalem, St. Augustine, Eusebius, and Theodoret, who even says the statue was a bronze one. Simon
broached many errors, which Noel Alexander enumerates and refutes (2). The principal ones were that
the world was created by angels; that when the soul leaves the body it enters into another body, which, if
true, says St. Iræneus (3), it would recollect all that happened when it inhabited the former body, for
memory, being a spiritual quality, it could not be separated from the soul. Another of his errors was one
which has been brought to light by the heretics of our own days, that man had no free will, and,
consequently, that good works are not necessary for salvation. Baronius and Fleury relate (4), that, by
force of magic spells, he one day caused the devil to elevate him in the air; but St. Peter and St. Paul being
present, and invoking the name of Jesus Christ, he fell down and broke both his legs. He was carried
away by his friends; but his corporeal and mental sufferings preyed so much on him, that, in despair, he
cast himself out of a high window; and thus perished the first heretic who ever disturbed the Church of
Christ (5). Basnage, who endeavours to prove that St. Peter never was in Rome, and never filled the
pontifical chair of that city, says that this is all a fabrication; but we have the testimony of St. Ambrose, St.
Isidore of Pelusium, St. Augustine, St. Maximus, St. Philastrius, St. Cyril of Jerusalem, Severus Sulpicius,
Theodoret, and many others, in our favour. We have, besides, a passage in Seutonius, which corroborates
their testimony, for he says (lib. VI., cap. xii), that, while Nero assisted at the public sports, a man
endeavoured to fly, but, after elevating himself for a while, he fell down, and the Emperors pavilion was
sprinkled with his blood.\\
(2) Nat. Alex. t. 5, in fin. Dis. 24.\\
(3) St. Iræneus, de Heresi. l. 2, c. 58.\\
(4) Baron. Ann. 35, n. 14, ad. 17; Fleury, His. Eccl. t. 1, l. 2, n. 23; St. Augus.; St. Joan. Chris.\\
(5) Baron, n. 17; Nat. Alex. t. 5, c. 11; Orsi, Istor. Eccl. l. 1, n. 20, and l. 2, n. 19; Berti. Brev. Histor. t. 1, c. 3.\\

\section{Menander}
Menander was a Samaritan likewise, and a disciple of Simon Magus; he made his appearance in the
year of our Lord 73. He announced himself a messenger from the " Unknown Power," for the salvation of
mankind. No one, according to him, could be saved, unless he was baptized in his name, and his baptism,
he said, was the true resurrection, so that his disciples would enjoy immortality even in this life (6).
Cardinal Orsi adds, that Menander was the first who invented the doctrine of "Eons," and that he taught
that Jesus Christ exercised human functions in appearance alone.

\section{Cerinthus}
Cerinthus was the next after Menander, but he began to broach his doctrine in the same year (7). His
errors can be reduced to four heads : he denied that God was the creator of the world; he asserted that the
law of Moses was necessary for salvation; he also taught that after the resurrection Jesus Christ would
establish a terrestrial kingdom in Jerusalem, where the just would spend a thousand years in the
enjoyment of every sensual pleasure; and, finally, he denied the divinity of Jesus Christ. The account
Bernini gives of his death is singular (8). The Apostle St. John, he says, met him going into a bath, when,
turning to those along with him, he said, let us hasten out of this, lest we be buried alive, and they had
scarcely gone outside when the whole building fell with a sudden crash, and the unfortunate Cerinthus
was overwhelmed in the ruins. One of the impious doctrines of this heretic was, that Jesus was a mere
man, born as all other men are, and that, when he was baptized in the river Jordan, Christ descended on
him, that is, a virtue or power, in form of a dove, or a spirit sent by God to fill him with knowledge, and
communicate it to mankind; but after Jesus had fulfilled his mission, by instructing mankind and
working miracles, he was deserted by Christ, who returned to heaven, and left him to darkness and
death. Alas ! what impiety men fall into when they desert the light of faith, and follow their own weak
imaginations.\\
(6) Fleury, loc. cit. n. 42; N. Alex. loc. cit. art. 2\\
(7) N. Alex. t. 5, c. 11, or. 5; Fleury, t. 1, L 2, n. 42; Berti, loc. cit. : Orsi, t. 1, l. 2, n. 43.\\
(8) Bernin. Istor. del Eresia, t. 1, c. 1; St. Iren. 1. 3, c. 4, de S.\\

\section{Ebion}
Ebion prided himself in being a disciple of St. Peter, and could even bear to hear St. Pauls name
mentioned. He admitted the sacrament of baptism; but in the consecration of the Eucharist he used
nothing but water in the chalice; he, however, consecrated the host in unleavened bread, and Eusebius
says he performed this every Sunday. According to St. Jerome, the baptism of the Ebionites was admitted
by the Catholics. He endeavoured to unite the Mosaic and Christian law, and admitted no part of the 
New Testament, unless the Gospel of St. Matthew, and even that mutilated, as he left out two chapters,
and altered the others in many places. The ancient writers say that St. John wrote his Gospel to refute the
errors of Ebion. The most impious of his blasphemies was, that Jesus Christ was the son of Joseph and
Mary, born as the rest of men are; that he was but a mere man, but that, on account of his great virtue, the
Almighty adopted him as his Son (9).

\section{Saturninus and Basilides}
Saturninus and Basilides were disciples of Menander, whose history we have already seen; and they
made some additions to the heresy of their master. Saturninus, a native of Antioch, taught, with
Menander, as Fleury tells us (10), that there was one only Father, unknown to all, who created the angels,
and that seven angels created the world and man. The God of the Jews, he said, was one of these
rebellious angels, and it was to destroy him that Christ appeared in the form of man, though he never
had a real body. He condemned matrimony and procreation as an invention of the devil. He attributed
the Prophecies partly to the angels, partly to the devil, and partly to the God of the Jews. He also said,
according to St. Augustine (Heres. iii), that the Supreme Virtue that is, the Sovereign Father having
created the angels, seven of them rebelled against him, created man, and for this reason : Seeing a celestial
light, they wished to retain it, but it vanished from them; and they then created man to resemble it,
saying, " Let us make man to the image and likeness." Man being thus created, was like a mere worm,
incapable of doing anything, till the Sovereign Virtue, pitying his image, placed in him a spark of himself,
and gave him life. This is the spark which, at the dissolution of the body, flies to heaven. Those of his sect
alone, he said, had this spark; all the others were deprived of it, and, consequently, were reprobate.\\
(9) N. Alex. loc. cit, art. 6; Fleury, loc. cit. n, 42. [N.B Fleury puts Ebion first, next Cerinthus, and lastly
Menander,]\\
(10) Fleury, n. 19.\\

Basilides, according to Fleury, was a native of Alexandria, and even exceeded Saturninus in fanaticism.
He said that the Father, whom he called Abrasax, produced Nous, that is, Intelligence; who produced
Logos, or the Word; the Word produced Phronesis, that is, Prudence; and Prudence, Sophia and
Dunamis, that is, Wisdom and Power. These created the angels, who formed the first heaven and other
angels; and these, in their turn, produced a second heaven, and so on, till there were three hundred and
sixty-five heavens produced, according to the number of days in the year. The God of the Jews, he said,
was the head of the second order of angels, and because he wished to rule all nations, the other princes
rose up against him, and, on that account, God sent his first-born, Nous, to free mankind from the
dominion of the angels who created the world. This Nous, who, according to him, was Jesus Christ, was
an incorporeal virtue, who put on whatever form pleased him. Hence, when the Jews wished to crucify
him, he took the form of Simon the Cyrenean, and gave his form to Simon, so that it was Simon, and not
Jesus, who was crucified. Jesus, at the same time, was laughing at the folly of the Jews, and afterwards
ascended invisibly to heaven. On that account, he said, we should not venerate the crucifix, otherwise we
would incur the danger of being subject to the angels who created the world. He broached many other
errors; but these are sufficient to show his fanaticism and impiety. Both Saturninus and Basilides fled
from martyrdom, and always cloaked their faith with this maxim " Know others, but let no one know
you." Cardinal Orsi says (11) they practised magic, and were addicted to every species of incontinence,
but that they were careful in avoiding observation. They promulgated their doctrines before Menander,
in the year 125; but, because they were disciples of his, we have mentioned them after him.\\
(11) Orsi, t. 2, I. 3, n. 23.\\

\section{The Nicholites}
The Nicholites admitted promiscuous intercourse with married and single, and, also, the use of meats
offered to idols. They also said that the Father of Jesus Christ was not the creator of the world. Among the
other foolish doctrines they held, was one, that darkness, uniting with the Holy Ghost, produced a matrix
or womb, which brought forth four Eons; that from these four Eons sprung the evil Eon, who created the
Gods, the angels, men, and seven demoniacal spirits, This heresy was of short duration; but some new
Nicholites sprung up afterwards in the Milanese territory, who were condemned by Pope Nicholas II.
The Nicholites called themselves disciples of Nicholas the Deacon, who, according to Noel Alexander,
was esteemed a heresiarch by St. Eusebius, St. Hilarian, and St. Jerome. However, Clement of Alexandria,
Eusebius, Theodoret, Baronius, St. Ignatius the Martyr, Orsi, St. Augustine, Fleury, and Berti, acquit him
of this charge (12).\\
(12) Nat. Alex. t. 5, diss. 9; Baron. An. 68, n. 9; Orsi, t. 1, n. 64; Fleury, t. 1, L 2, n. 21; Berti, loc. cit.\\

\chapter{Heresies of the Second Century}
\chapter{Heresies of the Third Century}
\chapter{Heresies of the Fourth Century}
\chapter{Heresies of the Fifth Century}
\end{document}